\documentclass[10pt,a4paper]{article}
\usepackage[utf8]{inputenc}
\usepackage{tikz}
\usepackage{pgfplots, pgfplotstable}
\usepackage{ae}
\usepackage[brazil]{babel}
\usepackage[vmargin=2cm,hmargin=2cm,columnsep=0.75cm]{geometry}
\usepackage{float,nonfloat}
\usepackage{graphicx,color}
\usepackage{subcaption}
\usepackage{amsmath}
\usepackage{verbatim}

\makeatletter
\let\@institution\empty
\def\institution#1{\def\@institution{#1}}
\renewcommand{\maketitle}{
    \begin{center}
        {\Large\bfseries\@title\par\medskip}
        {\large
            \begin{tabular}[t]{c}%
                \@author
        \end{tabular}\par\medskip}
        {\itshape\@institution\par}
        {\itshape\@date\par}
\end{center}}
\makeatother

\newcommand{\pixel}{\textit{pixel} }
\newcommand{\pixels}{\textit{pixels} }
\newcommand{\kernel}{\textit{kernel} }
\newcommand{\kernels}{\textit{kernels} }

\begin{document}
% ============================================================================

\title{MC920: Introdução ao Processamento de Imagem Digital\\Tarefa 11}
\author{
    \begin{minipage}{6cm}
        \centering
        Martin Ichilevici de Oliveira\\
        RA 118077
    \end{minipage}
    \and
    \begin{minipage}{6cm}
        \centering
        Rafael Almeida Erthal Hermano\\
        RA 121286
    \end{minipage}
}
\institution{Instituto de Computação, Universidade Estadual de Campinas}
\date{\today}

\maketitle

\section{Operações Básicas}

\subsection{Dilatação}
A dilatação tem como principal função expandir as fronteiras da região de foreground dos pixels da imagem, sendo assim, a área das regiões de foreground aumentam, enquanto buracos nessas regiões diminuem. A dilatação de uma imagem $X$ por um dado elemento estruturante $B$ é dada por:

\begin{equation}
    \delta_B(x) = max\{x_k, k \in B\}
\end{equation}

Portanto, em imagens binárias, se na janela do elemento estruturante existir pelo menos um elemento de foreground, o ponto será conseiderado de foreground.

\begin{figure}[!ht]
    \centering
    \begin{subfigure}[ht]{0.45\textwidth}
        \includegraphics[width=\textwidth]{src.jpg}
        \caption{Imagem original}
    \end{subfigure}
    \qquad
    \begin{subfigure}[ht]{0.45\textwidth}
        \includegraphics[width=\textwidth]{dilation.jpg}
        \caption{Imagem dilatada}
    \end{subfigure}
    \label{fig:dilation}
\end{figure}

\subsection{Erosão}
A erosão tem como principal função diminuir as fronteiras da região de foreground dos pixels da imagem, sendo assim, a área das regiões de foreground diminuem, enquanto buracos nessas regiões aumentam. A erosão de uma imagem $X$ por um dado elemento estruturante $B$ é dada por:

\begin{equation}
    \in_B(x) = min\{x_k, k \in B\}
\end{equation}

Portanto, em imagens binárias, se na janela do elemento estruturante existir pelo menos um elemento de background, o ponto será conseiderado de background.

\begin{figure}[!ht]
    \centering
    \begin{subfigure}[ht]{0.45\textwidth}
        \includegraphics[width=\textwidth]{src.jpg}
        \caption{Imagem original}
    \end{subfigure}
    \qquad
    \begin{subfigure}[ht]{0.45\textwidth}
        \includegraphics[width=\textwidth]{erosion.jpg}
        \caption{Imagem erodida}
    \end{subfigure}
    \label{fig:erosion}
\end{figure}

\subsection{Abertura}
A abertura tende a remover pixels de foreground em regiões de ponta de foreground. A abertura pode ser definida como a composição de uma dilatação com uma erosão:

\begin{equation}
    \gamma_B = \delta_B \circ \in_B
\end{equation}

\begin{figure}[!ht]
    \centering
    \begin{subfigure}[ht]{0.45\textwidth}
        \includegraphics[width=\textwidth]{src.jpg}
        \caption{Imagem original}
    \end{subfigure}
    \qquad
    \begin{subfigure}[ht]{0.45\textwidth}
        \includegraphics[width=\textwidth]{opening.jpg}
        \caption{Imagem aberta}
    \end{subfigure}
    \label{fig:opening}
\end{figure}

\subsection{Fechamento}
O fechamento tende a remover pixels de background em regiões de ponta de background. O fechamento pode ser definido como a composição de uma erosão com uma dilatação:

\begin{equation}
    \varphi_B = \in_B \circ \delta_B
\end{equation}

\begin{figure}[!ht]
    \centering
    \begin{subfigure}[ht]{0.45\textwidth}
        \includegraphics[width=\textwidth]{src.jpg}
        \caption{Imagem original}
    \end{subfigure}
    \qquad
    \begin{subfigure}[ht]{0.45\textwidth}
        \includegraphics[width=\textwidth]{closing.jpg}
        \caption{Imagem fechada}
    \end{subfigure}
    \label{fig:closing}
\end{figure}

\section{Dualidade}
A erosão não é o inverso da dilatação, somente em alguns casos de
erosão isso ocorre. Erosão e dilatação são duais no seguinte sentido:

\begin{equation}
    (\delta_{f,B})^{C} = \in_{f^{C},B^{C}}
\end{equation}

Isso significa que o complemento de uma erosão é o mesmo que uma dilatação do complemento da imagem pelo elemento estrutural refletido.

\section{Transformações tudo ou nada}
As transformações tudo ou nada varrem a imagem $X$ com um elemento estruturante $B$, se o elemento estruturante for igual à mascara na região da imagem, na posição do centro do elemento estruturante é colocado uma cor de foreground, caso contrário, é colocada a cor de background.

\subsection{Detecção de cantos}
Utilizando os seguintes 4 elementos estruturantes:

\begin{table}[h]
\begin{tabular}{|l|l|l|l|l|l|l|l|l|l|l|l|l|l|l|}
\cline{1-3} \cline{5-7} \cline{9-11} \cline{13-15}
  & 1 &   &  &   & 1 &   &  & 0 & 0 &   &  &   & 0 & 0 \\ \cline{1-3} \cline{5-7} \cline{9-11} \cline{13-15}
1 & 1 & 0 &  & 0 & 1 & 1 &  & 0 & 1 & 1 &  & 1 & 1 &   \\ \cline{1-3} \cline{5-7} \cline{9-11} \cline{13-15}
  & 0 & 0 &  & 0 & 0 &   &  &   & 1 &   &  &   & 1 &   \\ \cline{1-3} \cline{5-7} \cline{9-11} \cline{13-15}
\end{tabular}
\end{table}

Aplicando um OR com todas as transformações, obtemos todos os cantos da imagem.

\section{Imagens coloridas}

% ============================================================================


\begin{thebibliography}{99}
    \bibitem{livro} GONZALEZ, Rafael C.; WOODS, Richard E.. \textbf{Digital Image Processing}. 3. ed. Upper Saddle River, NJ, EUA: Prentice-hall, 2006.
    \bibitem{site} http://homepages.inf.ed.ac.uk/rbf/HIPR2/morops.htm.
\end{thebibliography}
\end{document}
