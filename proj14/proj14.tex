\documentclass[10pt,a4paper]{article}
\usepackage[utf8]{inputenc}
\usepackage{tikz}
\usepackage{pgfplots, pgfplotstable}
\usepackage{ae}
\usepackage[brazil]{babel}
\usepackage[vmargin=2cm,hmargin=2cm,columnsep=0.75cm]{geometry}
\usepackage{float,nonfloat}
\usepackage{graphicx,color}
\usepackage{subcaption}
\usepackage{amsmath}
\usepackage{verbatim}

\makeatletter
\let\@institution\empty
\def\institution#1{\def\@institution{#1}}
\renewcommand{\maketitle}{
    \begin{center}
        {\Large\bfseries\@title\par\medskip}
        {\large
            \begin{tabular}[t]{c}%
                \@author
        \end{tabular}\par\medskip}
        {\itshape\@institution\par}
        {\itshape\@date\par}
\end{center}}
\makeatother

\newcommand{\pixel}{\textit{pixel} }
\newcommand{\pixels}{\textit{pixels} }
\newcommand{\kernel}{\textit{kernel} }
\newcommand{\kernels}{\textit{kernels} }

\begin{document}
% ============================================================================

\title{MC920: Introdução ao Processamento de Imagem Digital\\Tarefa 14}
\author{
    \begin{minipage}{6cm}
        \centering
        Martin Ichilevici de Oliveira\\
        RA 118077
    \end{minipage}
    \and
    \begin{minipage}{6cm}
        \centering
        Rafael Almeida Erthal Hermano\\
        RA 121286
    \end{minipage}
}
\institution{Instituto de Computação, Universidade Estadual de Campinas}
\date{\today}

\maketitle

\section{Operador morfológico de coerência direcional}

\begin{figure}[!ht]
    \centering
    \begin{subfigure}[ht]{0.20\textwidth}
        \includegraphics[width=\textwidth]{1.jpg}
        \caption{Imagem original}
    \end{subfigure}
    \qquad
    \begin{subfigure}[ht]{0.20\textwidth}
        \includegraphics[width=\textwidth]{1_filtered_window_3.jpg}
        \caption{Filtrada com janela igual à 3}
    \end{subfigure}
    \qquad
    \begin{subfigure}[ht]{0.20\textwidth}
        \includegraphics[width=\textwidth]{1_filtered_window_5.jpg}
        \caption{Filtrada com janela igual à 5}
    \end{subfigure}
\end{figure}

\begin{figure}[!ht]
    \centering
    \begin{subfigure}[ht]{0.20\textwidth}
        \includegraphics[width=\textwidth]{2.jpg}
        \caption{Imagem original}
    \end{subfigure}
    \qquad
    \begin{subfigure}[ht]{0.20\textwidth}
        \includegraphics[width=\textwidth]{2_filtered_window_3.jpg}
        \caption{Filtrada com janela igual à 3}
    \end{subfigure}
    \qquad
    \begin{subfigure}[ht]{0.20\textwidth}
        \includegraphics[width=\textwidth]{2_filtered_window_5.jpg}
        \caption{Filtrada com janela igual à 5}
    \end{subfigure}
\end{figure}

\begin{figure}[!ht]
    \centering
    \begin{subfigure}[ht]{0.20\textwidth}
        \includegraphics[width=\textwidth]{3.jpg}
        \caption{Imagem original}
    \end{subfigure}
    \qquad
    \begin{subfigure}[ht]{0.20\textwidth}
        \includegraphics[width=\textwidth]{3_filtered_window_3.jpg}
        \caption{Filtrada com janela igual à 3}
    \end{subfigure}
    \qquad
    \begin{subfigure}[ht]{0.20\textwidth}
        \includegraphics[width=\textwidth]{3_filtered_window_5.jpg}
        \caption{Filtrada com janela igual à 5}
    \end{subfigure}
\end{figure}

\begin{figure}[!ht]
    \centering
    \begin{subfigure}[ht]{0.20\textwidth}
        \includegraphics[width=\textwidth]{4.jpg}
        \caption{Imagem original}
    \end{subfigure}
    \qquad
    \begin{subfigure}[ht]{0.20\textwidth}
        \includegraphics[width=\textwidth]{4_filtered_window_3.jpg}
        \caption{Filtrada com janela igual à 3}
    \end{subfigure}
    \qquad
    \begin{subfigure}[ht]{0.20\textwidth}
        \includegraphics[width=\textwidth]{4_filtered_window_5.jpg}
        \caption{Filtrada com janela igual à 5}
    \end{subfigure}
\end{figure}

\begin{thebibliography}{99}
    \bibitem{livro} GONZALEZ, Rafael C.; WOODS, Richard E.. \textbf{Digital Image Processing}. 3. ed. Upper Saddle River, NJ, EUA: Prentice-hall, 2006.
\end{thebibliography}
\end{document}
