\documentclass[10pt,a4paper]{article}
\usepackage[utf8]{inputenc}
\usepackage{tikz}
\usepackage{pgfplots, pgfplotstable}
\usepackage{ae}
\usepackage[brazil]{babel}
\usepackage[vmargin=2cm,hmargin=2cm,columnsep=0.75cm]{geometry}
\usepackage{float,nonfloat}
\usepackage{graphicx,color}
\usepackage{subcaption}
\usepackage{amsmath}
\usepackage{verbatim}

\makeatletter
\let\@institution\empty
\def\institution#1{\def\@institution{#1}}
\renewcommand{\maketitle}{
    \begin{center}
        {\Large\bfseries\@title\par\medskip}
        {\large
            \begin{tabular}[t]{c}%
                \@author
        \end{tabular}\par\medskip}
        {\itshape\@institution\par}
        {\itshape\@date\par}
\end{center}}
\makeatother

\newcommand{\pixel}{\textit{pixel} }
\newcommand{\pixels}{\textit{pixels} }
\newcommand{\kernel}{\textit{kernel} }
\newcommand{\kernels}{\textit{kernels} }

\begin{document}
% ============================================================================

\title{MC920: Introdução ao Processamento de Imagem Digital\\Tarefa 8}
\author{
    \begin{minipage}{6cm}
        \centering
        Martin Ichilevici de Oliveira\\
        RA 118077
    \end{minipage}
    \and
    \begin{minipage}{6cm}
        \centering
        Rafael Almeida Erthal Hermano\\
        RA 121286
    \end{minipage}
}
\institution{Instituto de Computação, Universidade Estadual de Campinas}
\date{\today}

\maketitle

% ============================================================================

\section{Critérios de fidelidade aplicados a filtragem de ruídos}
\subsection{Critérios de fidelidade}
\subsubsection{Erro total}
O error total mede o quadrado das diferenças entre os pontos originais e o resultado.

\begin{equation}
    e = \sum_{x = 0}^{M - 1} \sum_{y = 0}^{N - 1} [\hat{f}(x,y) - f(x,y)]^2
    \label{eq:error}
\end{equation}

\subsubsection{Erro médio quadrático}
O erro médio quadrático pode ser definido como:

\begin{equation}
    e_{rms} = \left[\frac{1}{MN} \sum_{x = 0}^{M - 1} \sum_{y = 0}^{N - 1} [\hat{f}(x,y) - f(x,y)]^2 \right]^{\frac{1}{2}}
    \label{eq:error_rms}
\end{equation}

\subsubsection{Relação sinal ruído}
A relação sinal ruído pode ser definida como:

\begin{equation}
    SNR_{ms} =\frac{\sum_{x = 0}^{M - 1} \sum_{y = 0}^{N - 1} \hat{f}(x,y)^{2} }{\sum_{x = 0}^{M - 1} \sum_{y = 0}^{N - 1} [\hat{f}(x,y) - f(x,y)]^2}
    \label{eq:snr}
\end{equation}

\subsection{Critérios de qualidade}
Para imagens cuja a finalidade é a observação pelo olho humano, o único método correto de avaliar a qualidade da imagem é a avaliação subjetiva \cite{article}, contudo, o método de índice de similaridade estrutural(SSIM), se propõem à de forma objetiva conseguir resproduzir os resultados subjetivos. O SSIM pode ser definido como:

\begin{equation}
    SSIM(X,Y) = [l(x,y)]^{\alpha} \cdot [c(x,y)]^{\beta} \cdot [s(x,y)]^{\gamma}
    \label{eq:ssim_trash}
\end{equation}

Onde $l(x,y)$ é a comparação da luminância, $c(x,y)$ comparação de contraste e $s(x,y)$ compara estruturas. Os expoentes $\alpha, \beta, \gamma$ são parâmetros para ponderar as importâncias de cada componente, e todos devem ser positivos.

\subsubsection*{Comparação da luminância}
Para comparação de luminância, devemos utilizar uma função que seja simétrica, limitada e possua um máximo único. A função usada é dada por:

\begin{equation}
    l(x,y) = \frac{2 \mu_x \mu_y + C_1}{\mu_x^2 + \mu_y^2 + C_1}
    \label{eq:lum}
\end{equation}

Onde, $\mu_x, \mu_y$ são as médias dos pixels nos eixos e $C_1$ é uma constante para evitar instabilidades quando $\mu_x^2 + \mu_y^2$ se aproxima de $0$.

\subsubsection*{Comparação da contraste}
A função de comparação de contraste é análoga à comparação de luminância. A função usada é dada por:

\begin{equation}
    l(x,y) = \frac{2 \sigma_x \sigma_y + C_2}{\sigma_x^2 + \sigma_y^2 + C_2}
    \label{eq:cont}
\end{equation}

Onde, $\sigma_x, \sigma_y$ é o desvio padrão dos valores dos pixels nos eixos.

\subsubsection*{Comparação da estruturas}
A função usada na comparação de estruturas é dada por:

\begin{equation}
    l(x,y) = \frac{2 \sigma_{xy} + C_3}{\sigma_x \sigma_y + C_3}
    \label{eq:estru}
\end{equation}

Onde $\sigma_{xy}$ é definida como:

\begin{equation}
    \sigma_{xy} = \frac{1}{N - 1} \sum_{i = 1}^{N}(x_i - \mu_x)(y_i - \mu_y )
    \label{eq:sigmaxy}
\end{equation}

\subsubsection*{Implementação utilizada}
Do trabalho \cite{article}, vamos definir os expoentes como sendo $\alpha = \beta = \gamma = 1$. Tendo assim a função SSIM como sendo:

\begin{equation}
    SSIM(X,Y) = \frac{(2 \mu_x \mu_y + C_1) (2 \sigma_{xy} + C_2)}{(\mu_x^2 + \mu_y^2 + C_1)(\sigma_x^2 + \sigma_y^2 + C_2)}
    \label{eq:ssim}
\end{equation}

\subsection{Comparação entre imagens filtradas}
Foram aplicados os ruídos, gaussiano e sal e pimenta em uma imagem e em seguida, foram aplicados os filtros gaussiano, da mediana e difusão anisotrópica.
Com os resultados das filtragens, foram calculados os erros total, médio quadrático, relação sinal ruído e o índice de similaridade estrutural \cite{article}. Com os resultados a seguir:

\begin{table}[!ht]
\begin{tabular}{|c|c|c|c|c|}
\hline
 & Total & Médio quadrático & Sinal Ruído & SSIM \\ \hline
 Filtro gaussiano sobre ruído gaussiano         &  &  &  &  \\ \hline
 Filtro gaussiano sobre ruído sal e pimenta     &  &  &  &  \\ \hline
 Filtro da mediana sobre ruído gaussiano        &  &  &  &  \\ \hline
 Filtro da mediana sobre ruído sal e pimenta    &  &  &  &  \\ \hline
 Difusão anisotrópica sobre ruído gaussiano     &  &  &  &  \\ \hline
 Difusão anisotrópica sobre ruído sal e pimenta &  &  &  &  \\ \hline
\end{tabular}
\end{table}

Para a difusão anisotrópica, foi realizada um \textit{grid search} variando o número de iterações, \textit{gamma} e o \textit{kappa}, os resultados do índice de similaridade estrutural para cada item foram plotados nos seguintes gráficos.

\begin{center}
\begin{tikzpicture}
    \begin{axis}[xlabel=Iterações, ylabel=SSIM]
    \addplot coordinates {};
    \end{axis}
\end{tikzpicture}
\end{center}

\begin{center}
\begin{tikzpicture}
    \begin{axis}[xlabel=Kappa, ylabel=SSIM]
    \addplot coordinates {};
    \end{axis}
\end{tikzpicture}
\end{center}

\begin{center}
\begin{tikzpicture}
    \begin{axis}[xlabel=Gamma, ylabel=SSIM]
    \addplot coordinates {};
    \end{axis}
\end{tikzpicture}
\end{center}

\begin{thebibliography}{99}
    \bibitem{livro} GONZALEZ, Rafael C.; WOODS, Richard E.. \textbf{Digital Image Processing}. 3. ed. Upper Saddle River, NJ, EUA: Prentice-hall, 2006.
    \bibitem{article} WANG, Z.; BOVIK, Alan C.;SHEIKH, Hamid R.; SIMONCELLI, Eero P.; \textbf{Image Quality Assessment: From error visibility to structural similarity}. IEEE Transactions on Image Processing, vol. 13, no. 4, 2004.
\end{thebibliography}

\end{document}
