\documentclass[10pt,a4paper]{article}
\usepackage[utf8]{inputenc}
\usepackage{tikz}
\usepackage{pgfplots, pgfplotstable}
\usepackage{ae}
\usepackage[brazil]{babel}
\usepackage[vmargin=2cm,hmargin=2cm,columnsep=0.75cm]{geometry}
\usepackage{float,nonfloat}
\usepackage{graphicx,color}
\usepackage{subcaption}
\usepackage{amsmath}
\usepackage{verbatim}
\usepackage{booktabs}

\makeatletter
\let\@institution\empty
\def\institution#1{\def\@institution{#1}}
\renewcommand{\maketitle}{
    \begin{center}
        {\Large\bfseries\@title\par\medskip}
        {\large
            \begin{tabular}[t]{c}%
                \@author
        \end{tabular}\par\medskip}
        {\itshape\@institution\par}
        {\itshape\@date\par}
\end{center}}
\makeatother

\newcommand{\img}[3]{
    \begin{figure}[!ht]
        \centering
        \begin{subfigure}[ht]{0.3\textwidth}
            \fbox{\includegraphics[width=\textwidth]{#1}}
            \caption{Imagem original}
        \end{subfigure}
        \qquad
        \begin{subfigure}[ht]{0.3\textwidth}
            \fbox{\includegraphics[width=\textwidth]{#2}}
            \caption{4-vizinhança}
        \end{subfigure}
        \qquad
        \begin{subfigure}[ht]{0.3\textwidth}
            \fbox{\includegraphics[width=\textwidth]{#3}}
            \caption{8-vizinhança}
        \end{subfigure}
        \caption{Imagens original e aplicar o filtro com 4- e 8-vizinhança}
    \end{figure}
}
\begin{document}
% ============================================================================

\title{MC920: Introdução ao Processamento de Imagem Digital\\Tarefa 13}
\author{
    \begin{minipage}{6cm}
        \centering
        Martin Ichilevici de Oliveira\\
        RA 118077
    \end{minipage}
    \and
    \begin{minipage}{6cm}
        \centering
        Rafael Almeida Erthal Hermano\\
        RA 121286
    \end{minipage}
}
\institution{Instituto de Computação, Universidade Estadual de Campinas}
\date{\today}

\maketitle

\section{Operações Básicas}

Definimos o operador $S_q$ como abaixo.

\begin{equation}
    S_q (X) = \bigcup_{n=0}^{\infty} \epsilon_{nB} / \gamma_B \circ \epsilon_{nB}
\end{equation}

Podemos usar tanto a 4-vizinhança como a 8-vizinhança.

\textbf{\color{red}A operação é homotópica ou não?}
\img{img.png}{img_4_final.jpg}{img_8_final.jpg}
\img{teste1a.jpg}{teste1a_4_final.jpg}{teste1a_8_final.jpg}
\img{teste1b.jpg}{teste1b_4_final.jpg}{teste1b_8_final.jpg}
\img{teste2a.jpg}{teste2a_4_final.jpg}{teste2a_8_final.jpg}
\img{teste2b.jpg}{teste2b_4_final.jpg}{teste2b_8_final.jpg}
\begin{thebibliography}{99}
    \bibitem{livro} GONZALEZ, Rafael C.; WOODS, Richard E.. \textbf{Digital Image Processing}. 3. ed. Upper Saddle River, NJ, EUA: Prentice-hall, 2006.
\end{thebibliography}
\end{document}
